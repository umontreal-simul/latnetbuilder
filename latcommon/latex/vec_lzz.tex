% Copyright (c) 2012 Richard Simard, Pierre L'Ecuyer, Université de Montréal.
% 
% This file is part of Lattice Builder.
% 
% Lattice Builder is free software: you can redistribute it and/or modify
% it under the terms of the GNU General Public License as published by
% the Free Software Foundation, either version 3 of the License, or
% (at your option) any later version.
% 
% Lattice Builder is distributed in the hope that it will be useful,
% but WITHOUT ANY WARRANTY; without even the implied warranty of
% MERCHANTABILITY or FITNESS FOR A PARTICULAR PURPOSE.  See the
% GNU General Public License for more details.
% 
% You should have received a copy of the GNU General Public License
% along with Lattice Builder.  If not, see <http://www.gnu.org/licenses/>.

\defmodule {vec\_lzz}
  
This class builds vectors of \texttt{long}'s and is very similar to some NTL
classes for vectors of integers. It has been added for compatibility with NTL in 
declaration of types and method calls. The macros used provide template-like
classes for dynamic-sized arrays of \texttt{long}'s.



%%%%%%%%%%%%%%%%%%%%%%%%%%%%%%%%%%%%%%%%%%%%%%%%%%%%%%%%%%%%%
\bigskip\hrule
\code \hide
#ifndef VEC_LZZ_H
#define VEC_LZZ_H
\endhide
#include "NTL/vector.h"
#include "NTL/vec_lzz_p.h"

typedef long Long; 
\endcode
\tab 
   Definition of type \texttt{Long}.
  \hrichard{Cette d\'efinition est-elle vraiment utile?}.
\endtab
\code
          
NTL_vector_decl(Long,vec_zz)
\endcode
 \tab This macro declares a class \texttt{vec\_zz} of vectors of \texttt{Long}'s.
The declaration  \texttt{vec\_zz v;} creates a zero-length vector. The statement
\texttt{v.SetLength(n);} grows this vector to length $n$.
This causes space to be allocated for $n$ elements.
The current length of a vector is obtained by calling \texttt{v.length()}.
\endtab
\code

NTL_io_vector_decl(Long,vec_zz)
\endcode
 \tab This macro allows the use of the input and output operators
 $\scriptstyle{<<}$ and $\scriptstyle{>>}$ for type  \texttt{vec\_zz}.
\endtab
\code

NTL_eq_vector_decl(Long,vec_zz)
\endcode
 \tab This macro allows the use of the equality operators $==$ and $!=$ for type 
  \texttt{vec\_zz}.
\endtab
\code

void conv(NTL::vec_zz_p & Vp, const vec_zz & V);
\endcode
 \tab Converts \texttt{V} into \texttt{Vp}.
\endtab
\code

// typedef vec_zz vec_long; 

\hide
#endif
\endhide
\endcode
