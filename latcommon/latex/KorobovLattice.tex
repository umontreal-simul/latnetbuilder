% Copyright (c) 2012 Richard Simard, Pierre L'Ecuyer, Université de Montréal.
% 
% This file is part of Lattice Builder.
% 
% Lattice Builder is free software: you can redistribute it and/or modify
% it under the terms of the GNU General Public License as published by
% the Free Software Foundation, either version 3 of the License, or
% (at your option) any later version.
% 
% Lattice Builder is distributed in the hope that it will be useful,
% but WITHOUT ANY WARRANTY; without even the implied warranty of
% MERCHANTABILITY or FITNESS FOR A PARTICULAR PURPOSE.  See the
% GNU General Public License for more details.
% 
% You should have received a copy of the GNU General Public License
% along with Lattice Builder.  If not, see <http://www.gnu.org/licenses/>.

\defmodule {KorobovLattice}
\hpierre{Reprogrammer \texttt{incDim} de fa\c con efficace comme dans
 \texttt{MRGLattice}}

This class implements lattice bases built from a Korobov lattice rule. 
For a given $a$, a Korobov lattice basis is formed as follows:
\[
\mathbf{b_1} = (1, a, a^2, \ldots, a^{d-1}),\quad
 \mathbf{b_2} = (0, n, 0, \ldots, 0),\quad  \ldots,\quad
  \mathbf{b_d} = (0, \ldots, 0, n).
\]


%%%%%%%%%%%%%%%%%%%%%%%%%%%%%%%%%%%%%%%%%%%%%%%%%%%%%%%%%%
\bigskip\hrule
\code\hide
#ifndef KOROBOVLATTICE_H
#define KOROBOVLATTICE_H
\endhide
#include "latcommon/Types.h"
#include "latcommon/Const.h"
#include "latcommon/IntLattice.h"


namespace LatCommon {

class KorobovLattice: public IntLattice {
public:

   KorobovLattice (const MScal & n, const MScal & a, int maxDim,
                   NormType norm = L2NORM);
\endcode
\tabb
  Constructs a Korobov lattice with $n$ points, maximal dimension
\texttt{maxDim} using the norm \texttt{norm}.
\endtabb
\code

   KorobovLattice (const MScal & n, const MScal & a, int dim, int t,
                   NormType norm = L2NORM);
\endcode
\tabb
Constructor. Same as above, except the lattice is formed as follow:
\[
\mathbf{b_1} = (a^t, a^{t+1}, a^{t+2}, \ldots, a^{t+d-1}),\qquad 
\mathbf{b_2} = (0, n, 0, \ldots, 0),\qquad  \ldots,\qquad 
 \mathbf{b_d} = (0, \ldots, 0, n).
\]
\endtabb
\code

   KorobovLattice (const KorobovLattice & Lat);
\endcode
\tabb
Copy constructor.
\endtabb
\code

   KorobovLattice & operator= (const KorobovLattice & Lat);
\endcode
\tabb
Assigns \texttt{Lat} to this object.
\endtabb
\code

   ~KorobovLattice();
\endcode
\tabb
Destructor.
\endtabb
\code

   std::string toStringCoef() const;
\endcode
\tabb
Returns the multiplier $a$ as a string.
\endtabb
\code

   void buildBasis (int d);
\endcode
\tabb
Builds the basis in dimension $d$.
\endtabb
\code

   void incDim();
\endcode
\tabb
   Increments the dimension of the basis by 1.
\endtabb
\code

   void incDimSlow();
\endcode
\tabb
   Increments the dimension of the basis by 1 by rebuilding the basis
from scratch. This is very slow. It can be used for verification of the
fast \texttt{incDim} method above.
\endtabb
\code


protected:

   MScal m_a;
\endcode
\tabb
The multiplier of the Korobov lattice rule.
\endtabb
\code

   int m_shift;
\endcode
\tabb
The shift applied to the lattice rule.
\endtabb
\code

   void init();
\endcode
\tabb
Initialization.
\endtabb
\code

};

}
\hide
#endif
\endhide
\endcode
