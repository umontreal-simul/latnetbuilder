% Copyright (c) 2012 Richard Simard, Pierre L'Ecuyer, Université de Montréal.
% 
% This file is part of Lattice Builder.
% 
% Lattice Builder is free software: you can redistribute it and/or modify
% it under the terms of the GNU General Public License as published by
% the Free Software Foundation, either version 3 of the License, or
% (at your option) any later version.
% 
% Lattice Builder is distributed in the hope that it will be useful,
% but WITHOUT ANY WARRANTY; without even the implied warranty of
% MERCHANTABILITY or FITNESS FOR A PARTICULAR PURPOSE.  See the
% GNU General Public License for more details.
% 
% You should have received a copy of the GNU General Public License
% along with Lattice Builder.  If not, see <http://www.gnu.org/licenses/>.

\defmodule {Reducer}

For a given lattice, this class implements methods to reduce its basis in the
sense of Minkowski and to find the shortest non-zero vector of the lattice
using pre-reductions and a branch-and-bound (BB) algorithm \cite{rLEC97c}.
It also implements the method of  Lenstra, Lenstra and Lovasz (LLL) \cite{mLEN82a}
as well as  the method of Dieter \cite{rDIE75a} to reduce a lattice basis.
The reduction algorithms in this class use both the primal and the dual lattices,
so both lattices must be defined.

%%%%%%%%%%%%%%%%%%%%%%%%%%%%%%%%%%%%%%%%%%%%%%%%%%%%%%%%%%%%%
\bigskip\hrule
\code\hide
#ifndef REDUCER_H
#define REDUCER_H
\endhide 
#include "latcommon/Types.h"
#include "latcommon/Const.h"
#include "latcommon/UtilLC.h"
#include "latcommon/Base.h"
#include "latcommon/IntLattice.h"
#include <fstream>
#include <sstream>
#include <vector>


namespace LatCommon {

class Reducer {
public:

   static const long SHORT_DIET = 1000;
\endcode
\tabb
   Whenever the number of nodes in the branch-and-bound tree exceeds
  \texttt{SHORT\_DIET} in the method \texttt{ShortestVector},
   \texttt{PreRedDieterSV}
   is automatically set to \texttt{true} for the next call;
   otherwise it is set to \texttt{false}.
%   The default value is 1000.
\endtabb
\code

   static const long SHORT_LLL = 1000;
\endcode
\tabb
   Whenever the number of nodes in the branch-and-bound tree exceeds
   \texttt{SHORT\_LLL} in the  method \texttt{ShortestVector},
   \texttt{PreRedLLLSV} is automatically set to \texttt{true} for the next call;
   otherwise it is set to \texttt{false}.
%   The default value is 1000.
\endtabb
\code

   static const long MINK_LLL = 500000;
\endcode
\tabb
   Whenever the number of nodes in the branch-and-bound tree exceeds
  \texttt{MINK\_LLL} in the  method \texttt{reduct\-Minkowski},
   \texttt{PreRedLLLRM} is automatically set to \texttt{true} for the
   next call; otherwise it is set to \texttt{false}.
\endtabb
\code

   static const long MAX_PRE_RED = 1000000;
\endcode
\tabb
   Maximum number of transformations in the method \texttt{PreRedDieter}.
   After \texttt{MAX\_PRE\_RED} successful transformations have been performed,
   the prereduction is stopped.
\endtabb
\code

   static long maxNodesBB;
\endcode
\tabb
   The maximum number of nodes % that we are ready to accept
   in the branch-and-bound tree when calling \texttt{ShortestVector}
   or \texttt{reductMinkowski}.  When this number is exceeded, the method
   aborts and returns \texttt{false}.
\endtabb
\code

   static bool PreRedDieterSV;
   static bool PreRedLLLSV;
   static bool PreRedLLLRM;
\endcode
\tabb
These boolean variables indicate which type of pre-reduction
   is to be performed for \texttt{ShortestVector} (SV) and for
   \texttt{reductMinkowski} (RM).
   \texttt{Dieter} means the pairwise pre-reduction as in the method
   \texttt{PreRedDieter}.
   \texttt{LLL} means the LLL reduction of Lenstra, Lenstra, and Lov\'asz.
   The variable \texttt{PreRedDieterSV} is originally set to \texttt{true}
   and the two others are originally set to \texttt{false}.
   These variables are reset automatically depending on the thresholds
   \texttt{MinkLLL, ShortDiet, ShortLLL} as explained above.
\endtabb
\code

   Reducer (IntLattice & lat);
\endcode
\tabb
Constructor. Initializes the reducer to work on lattice \texttt{lat}.
\endtabb
\code

   Reducer (const Reducer & red);
\endcode
\tabb
Copy constructor.
\endtabb
\code

   ~Reducer ();
\endcode
\tabb
Destructor.
\endtabb
\code

   Reducer & operator= (const Reducer & red);
\endcode
\tabb
Assignment operator.
\endtabb
\code

   void copy (const Reducer & red);
\endcode
\tabb
Copies the reducer \texttt{red} into this object. \richard{Encore utile?}
\endtabb
\code

   void transformStage3 (std::vector<long> & z, int & k);
\endcode
\tabb
  Method used in \texttt{reductMinkowski} to perform a transformation
  of stage 3 described in \cite{rAFF85a}. Also used in \texttt{ShortestVector}.
  Assumes that $\sum_{i=1}^t z_i V_i$ is a short vector that will enter
  the basis.  Tries to reduce some vectors by looking for indices $i < j$
  such that $|z_j| > 1$ and $q=\lfloor z_i/z_j\rfloor \not=0$,
  and adding $q V_i$ to $V_j$ when this happens.
  Returns in $k$ the last index $j$ such that $|z_j|=1$.
\endtabb
\code

   bool calculCholeski (RVect & DC2, RMat & C0);
\endcode
\tabb Finds a Choleski decomposition of the basis. Returns in \texttt{C0}
   the elements of the upper triangular matrix of the Choleski decomposition
   that are above the diagonal. Returns in \texttt{DC2} the
  squared elements of the diagonal. 
\endtabb
\code

   bool tryZ  (int j, int i, int Stage, bool & smaller, Base & WTemp);
\endcode
\tabb Tries to find shorter vectors in \texttt{reductMinkowski}.
\endtabb
\code

   bool tryZ0 (int j, bool & smaller);
\endcode
\tabb Tries to find a shorter vector in \texttt{shortestVector}.
\endtabb
\code

   bool shortestVector (NormType norm);
\endcode
\tabb
  Computes the shortest non-zero vector of this lattice with respect to norm
  \texttt{norm} using branch-and-bound and the algorithm described in
   \cite{rLEC97c}.
  The \texttt{Norm} member of this object  will be changed to \texttt{norm}.
   If \texttt{MaxNodesBB} is exceeded during {one} of the branch-and-bounds,
   the method aborts and returns \texttt{false}.
   Otherwise, it returns \texttt{true}. Uses the pre-reduction algorithms
   of Dieter and of Lenstra-Lenstra-Lovasz.
\endtabb
\code

   bool shortestVectorDieter (NormType norm);
\endcode
\tabb
Similar to \texttt{ShortestVector} but uses the algorithm of
  Dieter \cite{rDIE75a,rKNU98a}.
\endtabb
\code

   bool redBB (int i, int d, int Stage, bool & smaller);
\endcode
\tabb
  Tries to shorten the vectors of the primal basis using branch-and-bound,
  in \texttt{reductMinkowski}.
\endtabb
\code

   bool redBB0 (NormType norm);
\endcode
\tabb  Tries to shorten the smallest vector of the primal basis 
  using branch-and-bound, in \texttt{ShortestVector}.
\endtabb
\code

   void preRedDieter (int d);
\endcode
\tabb
Performs the reductions of the preceding two methods
  using cyclically all values of $i$ (only for $i > d$ in the latter case)
  and stops after either \texttt{MaxPreRed} successful transformations have been
  achieved or no further reduction is possible.
  Always use the Euclidean norm.
\endtabb
\code

   bool redDieter (NormType norm);
\endcode
\tabb
Finds the shortest non-zero vector using norm \texttt{norm}.
%%  Does not modify the basis.
%% Stops and returns \texttt{false} if not finished after examining possib.
 Returns \texttt{true} upon success. %%, and returns length in Lmin.
 Uses the algorithm of Dieter \cite{rDIE75a} given in Knuth \cite{rKNU98a}.
\endtabb
\code

   void redLLL (double fact, long maxcpt, int dim);
\endcode
\tabb
Performs a LLL (Lenstra-Lenstra-Lovasz) basis reduction up to dimension
   \texttt{dim} with coefficient \texttt{fact}, which must be smaller than 1.
   If \texttt{fact} is closer to 1, the basis will be
   (typically) ``more reduced'', but that will require more work.
   The reduction algorithm is applied until \texttt{maxcpt} successful
   transformations have been done. Always uses the Euclidean norm.
\endtabb
\code

   bool reductMinkowski (int d);
\endcode
\tabb
Reduces the current basis to a Minkowski reduced basis
    with respect to the Euclidean norm, assuming
   that the first $d$ vectors are already reduced and sorted.
   If \texttt{MaxNodesBB} is exceeded during {one} of the branch-and-bound step,
   the method aborts and returns \texttt{false}.
   Otherwise it returns \texttt{true}, the basis is reduced and
    sorted by vector lengths (the shortest vector is \texttt{V[1]} and
   the longest is \texttt{V[Dim]}).
%   This method does not care about numerical imprecision due to the
%    (64-bit) floating-point representation.
%    In this sense, the results are not 100\% reliable.
\endtabb
\code

   void pairwiseRedPrimal (int i, int d);
\endcode
\tabb
Performs pairwise reductions.
  This method tries to reduce each basis vector with
  index larger than $d$ and distinct from $i$ by adding to it a
  multiple of the $i$-th vector.
  Always uses the Euclidean norm.
\endtabb
\code

   void pairwiseRedDual (int i);
\endcode
\tabb
Performs pairwise reductions, trying to reduce every other vector of
  the {\em dual\/} basis by adding multiples of the $i$-th vector.
  That may change the $i$-th vector in the primal basis.
  Each such dual reduction is actually performed only if that does not
  increase the length of vector $i$ in the primal basis.
  Always uses the Euclidean norm.
\endtabb
\code

   NScal getMinLength () \hide {
      if (m_lat->getNorm() == L2NORM)
         return sqrt(m_lMin2);
      else return m_lMin; } \endhide
\endcode
\tabb
Returns the length of the shortest basis vector in the lattice.
\endtabb
\code

   void setBoundL2 (const NVect & V, int dim1, int dim2);
\endcode
\tabb
  Sets the lower bound on the square length of the shortest vector in
  dimension $i$ to $V[i]$,  for $i$ going from \texttt{dim1} to \texttt{dim2}.
\endtabb
\code

   void trace (char *mess);
\endcode
\tabb
Debug function that print the primal and dual bases.
\endtabb
\ifdetailed
\code


private:

   IntLattice* m_lat;
\endcode
\tabb
Lattice on which the reduction will be performed.
\endtabb
\code

   void permuteGramVD (int i, int j, int n);
\endcode
\tabb
Permutes the $i^{th}$ and the $j^{th}$ line, and the $i^{th}$ and the $j^{th}$ column
of the scalar product's matrix.
\endtabb
\code

   void calculCholeski2LLL (int n, int j);
\endcode
\tabb
Recalculates the first $n$ entries of the $j^{th}$ column of the Choleski's matrix
of order 2.
\endtabb
\code

   void calculCholeski2Ele (int i, int j);
\endcode
\tabb
Recalculates the entry ($i$, $j$) of the Choleski's matrix of order 2.
\endtabb
\code

   void miseAJourGramVD (int j);
\endcode
\tabb
\endtabb
\code

   void calculGramVD ();
\endcode
\tabb
\endtabb
\code

   void reductionFaible (int i, int j);
\endcode
\tabb
\endtabb
\code

   NVect m_BoundL2;
\endcode
\tabb  
 Lower bound on the length squared of the shortest vector in each dimension.
 If any vector of the lattice is shorter than this bound, we stop the reduction
immediately and reject this lattice since its shortest vector will be
even smaller.
\endtabb
\code

   BScal m_bs;
   BVect m_bv;            // Saves shorter vector in primal basis
   BVect m_bw;            // Saves shorter vector in dual basis

   NScal m_lMin, m_lMin2;
   NScal m_ns;
   NVect m_nv; 

   RScal m_rs;
   RVect m_zLR, m_n2, m_dc2;
   RMat m_c0, m_c2, m_cho2, m_gramVD;
   int *m_IC;             // Indices in Cholesky 
   
   std::vector<long> m_zLI;
   std::vector<long> m_zShort;
   long m_countNodes;     // Counts number of nodes in the branch-and-bound tree
   long m_countDieter;    // Counts number of attempts since last successful
                          // Dieter transformation
   long m_cpt;            // Counts number of successes in pre-reduction
                          // transformations
   bool m_foundZero;      // = true -> the zero vector has been found
// long m_BoundL2Count;   // Number of cases for which the reduction stops
                          // because any vector is shorter than L2 Bound.
\endcode
\tabb
  Work variables.
\endtabb
\fi
\code
};

}

\hide
#endif // REDUCER_H
\endhide
\endcode
