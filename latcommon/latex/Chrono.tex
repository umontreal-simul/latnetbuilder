% Copyright (c) 2012 Richard Simard, Pierre L'Ecuyer, Université de Montréal.
% 
% This file is part of Lattice Builder.
% 
% Lattice Builder is free software: you can redistribute it and/or modify
% it under the terms of the GNU General Public License as published by
% the Free Software Foundation, either version 3 of the License, or
% (at your option) any later version.
% 
% Lattice Builder is distributed in the hope that it will be useful,
% but WITHOUT ANY WARRANTY; without even the implied warranty of
% MERCHANTABILITY or FITNESS FOR A PARTICULAR PURPOSE.  See the
% GNU General Public License for more details.
% 
% You should have received a copy of the GNU General Public License
% along with Lattice Builder.  If not, see <http://www.gnu.org/licenses/>.

\defmodule{Chrono}

This class acts as an interface to the system clock to compute the
CPU time used by parts of a program.
  Even though the ANSI/ISO macro \texttt{CLOCKS\_PER\_SEC = 1000000} 
  is the number of clock ticks per second for the value
  returned by the ANSI-C standard \texttt{clock} function 
  (so this function returns the
  number of microseconds), on some systems where the 32-bit type \texttt{long} 
  is used to measure time, the value returned by \texttt{clock}
  wraps around to negative values after about 36 minutes.
  On some other systems where time is measured using the 32-bit type
  \texttt{unsigned long}, the clock may wrap around to 0 after about
   72 minutes.
  When the macro \texttt{USE\_ANSI\_CLOCK} is undefined, 
  a non-ANSI-C clock is used. 
  On Linux-Unix systems, it calls the POSIX
  function \texttt{times} to get the CPU time used by a program.
\hpierre{Is this true for both Linux and Windows?}
\hrichard{Je ne crois pas que MicroMou respecte le standard POSIX, qui est
 sp\'ecifique aux syst\`emes Unix ou Linux.}
  On a Windows platform (when the macro \texttt{HAVE\_WINDOWS\_H} is defined),
  the Windows function \texttt{GetProcessTimes} will be used to measure
  the CPU time used by programs.


Every object \texttt{Chrono} acts as an independent 
\emph{stopwatch}.  Several such stopwatchs can run at any given time.
An object of type \texttt{Chrono} must be declared 
for each of them.
The method \texttt{init} resets the stopwatch to zero,
\texttt{val} returns its current reading,
and \texttt{write} writes this reading to the current output.
The returned value includes part of the execution time of the functions
from class \texttt{Chrono}.
The \texttt{TimeFormat} allows one to choose the kind of 
time units that are used.  

Below is an example of how the functions may be used.
A stopwatch named \texttt{timer} is declared and created.
After 2.1 seconds of CPU time have been consumed, the stopwatch is read and
reset. Then, after an additional 330 seconds (or 5.5 minutes) of CPU time
the stopwatch is read again, printed to the output and deleted.
%
 \begin{verse}{\tt
  Chrono timer; \\
\hskip 1.0cm   \vdots 
\hskip 1.0cm  (\emph{suppose 2.1 CPU seconds are used here}.)\\[6pt]
  double t = timer.val (Chrono::SEC); \qquad   // Here, t = 2.1 \\
  timer.init(); \\
\hskip 1.0cm  \vdots
\hskip 1.0cm (\emph{suppose 330 CPU seconds are used here}.) \\[10pt]
  t = timer.val (Chrono::MIN); \qquad    // Here, t = 5.5 \\
  timer.write (Chrono::HMS);  \qquad     // Prints: 00:05:30.00 \\
 }\end{verse}

% \newpage

%%%%%%%%%%%%%%%%%%%%%%%%%%%%%%%%%%%%%%%%%%%%%%%%%%%%%%%%%%%%%

\bigskip\hrule
\code\hide
#ifndef CHRONO_H
#define CHRONO_H 
\endhide
#include <string>

#undef USE_ANSI_CLOCK
\endcode
  \tab
  On a MS-Windows platform,
  %(when the macro \texttt{HAVE\_WINDOWS\_H} is defined),
  the MS-Windows function \texttt{GetProcessTimes} will be used to measure
  the CPU time used by programs.
  On Linux$|$Unix platforms, if the macro
  \texttt{USE\_ANSI\_CLOCK} is defined, the timers % in module \texttt{chrono}
  will call the ANSI C  \texttt{clock} function.
  When \texttt{USE\_ANSI\_CLOCK} is left undefined, class
  \texttt{Chrono} gets the CPU time used by a program via an 
  alternate non-ANSI C timer 
  based on the POSIX (The Portable Operating System Interface)
  function \texttt{times}, assuming this function is available. The POSIX
  standard is described in the IEEE Std 1003.1-2001 document (see 
  The Open Group web site at
  \url{http://www.opengroup.org/onlinepubs/007904975/toc.htm}).
 \endtab
\code


namespace LatCommon {

class Chrono {
public:

   enum TimeFormat { SEC, MIN, HOURS, DAYS, HMS };
\endcode
 \tabb
  Types of units in which the time on a \texttt{Chrono} can be 
  read or printed: in seconds (\texttt{SEC}), minutes (\texttt{MIN}),
  hours (\texttt{HOUR}), days
  (\texttt{DAYS}), or in the \texttt{HH:MM:SS.xx} format, with hours, 
  minutes, seconds and hundreths of a second (\texttt{HMS}).
 \endtabb
\code

   Chrono();
\endcode
  \tabb Constructor for a stopwatch; initializes it to zero.
   One may reinitialize it later by calling \texttt{init}.
  \endtabb
\code

   ~Chrono() \hide {} \endhide
\endcode
\tabb
Destructor.
\endtabb
\code

   void init ();
\endcode
  \tabb
  Initializes this stopwatch to zero.
  \endtabb
\code

   double val (TimeFormat unit);
\endcode
  \tabb
  Returns the CPU time measured by this \texttt{Chrono} since the last call to
  \texttt{init()}. The parameter \texttt{unit} specifies the time unit.
  Restriction: \texttt{unit = HMS} is not allowed here; 
  it will cause an error.
  \endtabb
\code

   void write (TimeFormat unit);
\endcode
 \tabb
  Prints, on standard output, the CPU time measured by this 
  \texttt{Chrono} since its last  call to \texttt{init()}.
  The parameter \texttt{unit} specifies the time unit.
 \endtabb
\code

   std::string toString();
\endcode
 \tabb
  Returns as a string the CPU time measured by this \texttt{Chrono}
  since its last call to \texttt{init()}. The time format used is \texttt{HMS}.
 \endtabb
\code

   bool timeOver (double limit);
\endcode
 \tabb
 Returns \texttt{true} if this \texttt{Chrono} has reached the 
 time \texttt{limit} (in seconds), otherwise returns \texttt{false}.
 \endtabb
\code


private:

   unsigned long microsec;         // microseconds
   unsigned long second;           // seconds

   void heure();
};

}
\hide
#endif
\endhide
\endcode
